\documentclass{article}

\usepackage[utf8]{inputenc}
\usepackage[T1]{fontenc}
\usepackage[french]{babel}
\usepackage{amsmath}
\usepackage{tikz}
\usepackage{lipsum}
\usepackage[margin=3cm]{geometry}
\usepackage{listings}
\usepackage{csquotes}
\usepackage{subcaption,booktabs}

\usepackage{xcolor}
\usepackage{listings}
\definecolor{dkgreen}{rgb}{0,0.6,0}
\definecolor{ltblue}{rgb}{0,0.4,0.4}
\definecolor{dkviolet}{rgb}{0.3,0,0.5}

% lstlisting coq style (inspired from a file of Assia Mahboubi)
\lstdefinelanguage{Coq}{ 
    % Anything betweeen $ becomes LaTeX math mode
    mathescape=true,
    % Comments may or not include Latex commands
    texcl=false, 
    % Vernacular commands
    morekeywords=[1]{Section, Module, End, Require, Import, Export,
        Variable, Variables, Parameter, Parameters, Axiom, Hypothesis,
        Hypotheses, Notation, Local, Tactic, Reserved, Scope, Open, Close,
        Bind, Delimit, Definition, Let, Ltac, Fixpoint, CoFixpoint, Add,
        Morphism, Relation, Implicit, Arguments, Unset, Contextual,
        Strict, Prenex, Implicits, Inductive, CoInductive, Record,
        Structure, Canonical, Coercion, Context, Class, Global, Instance,
        Program, Infix, Theorem, Lemma, Corollary, Proposition, Fact,
        Remark, Example, Proof, Goal, Save, Qed, Defined, Hint, Resolve,
        Rewrite, View, Search, Show, Print, Printing, All, Eval, Check,
        Projections, inside, outside, Def},
    % Gallina
    morekeywords=[2]{forall, exists, exists2, fun, fix, cofix, struct,
        match, with, end, as, in, return, let, if, is, then, else, for, of,
        nosimpl, when},
    % Sorts
    morekeywords=[3]{Type, Prop, Set, true, false, option},
    % Various tactics, some are std Coq subsumed by ssr, for the manual purpose
    morekeywords=[4]{pose, set, move, case, elim, apply, clear, hnf,
        intro, intros, generalize, rename, pattern, after, destruct,
        induction, using, refine, inversion, injection, rewrite, congr,
        unlock, compute, ring, field, fourier, replace, fold, unfold,
        change, cutrewrite, simpl, have, suff, wlog, suffices, without,
        loss, nat_norm, assert, cut, trivial, revert, bool_congr, nat_congr,
        symmetry, transitivity, auto, split, left, right, autorewrite},
    % Terminators
    morekeywords=[5]{by, done, exact, reflexivity, tauto, romega, omega,
        assumption, solve, contradiction, discriminate},
    % Control
    morekeywords=[6]{do, last, first, try, idtac, repeat},
    % Comments delimiters, we do turn this off for the manual
    morecomment=[s]{(*}{*)},
    % Spaces are not displayed as a special character
    showstringspaces=false,
    % String delimiters
    morestring=[b]",
    morestring=[d],
    % Size of tabulations
    tabsize=3,
    % Enables ASCII chars 128 to 255
    extendedchars=false,
    % Case sensitivity
    sensitive=true,
    % Automatic breaking of long lines
    breaklines=false,
    % Default style fors listings
    basicstyle=\small,
    % Position of captions is bottom
    captionpos=b,
    % flexible columns
    columns=[l]flexible,
    % Style for (listings') identifiers
    identifierstyle={\ttfamily\color{black}},
    % Style for declaration keywords
    keywordstyle=[1]{\ttfamily\color{dkviolet}},
    % Style for gallina keywords
    keywordstyle=[2]{\ttfamily\color{dkgreen}},
    % Style for sorts keywords
    keywordstyle=[3]{\ttfamily\color{ltblue}},
    % Style for tactics keywords
    keywordstyle=[4]{\ttfamily\color{dkblue}},
    % Style for terminators keywords
    keywordstyle=[5]{\ttfamily\color{dkred}},
    %Style for iterators
    %keywordstyle=[6]{\ttfamily\color{dkpink}},
    % Style for strings
    stringstyle=\ttfamily,
    % Style for comments
    commentstyle={\ttfamily\color{dkgreen}},
    %moredelim=**[is][\ttfamily\color{red}]{/&}{&/},
    literate=
    {\\forall}{{\color{dkgreen}{$\forall\;$}}}1
    {\\exists}{{$\exists\;$}}1
    {<-}{{$\leftarrow\;$}}1
    {=>}{{$\Rightarrow\;$}}1
    {==}{{\code{==}\;}}1
    {==>}{{\code{==>}\;}}1
    %    {:>}{{\code{:>}\;}}1
    {->}{{$\rightarrow\;$}}1
    {<->}{{$\leftrightarrow\;$}}1
    {<==}{{$\leq\;$}}1
    {\#}{{$^\star$}}1 
    {\\o}{{$\circ\;$}}1 
    {\@}{{$\cdot$}}1 
    {\/\\}{{$\wedge\;$}}1
    {\\\/}{{$\vee\;$}}1
    {++}{{\code{++}}}1
    {~}{{\ }}1
    {\@\@}{{$@$}}1
    {\\mapsto}{{$\mapsto\;$}}1
    {\\hline}{{\rule{\linewidth}{0.5pt}}}1
    %
}[keywords,comments,strings]




% LTeX: SETTINGS enabled=false

\usepackage{hyperref}
\hypersetup{
    colorlinks,
    citecolor=black,
    filecolor=black,
    linkcolor=black,
    urlcolor=blue
}

\title{Stage L3: Formalisation des dictionnaire en Coq}
\author{Valeran MAYTIE}
\date{Juillet 2023}

% LTeX: SETTINGS enabled=true language=fr

\begin{document}
  \maketitle

  \section{Structure d'accueil}

  Notre stage se déroule au bâtiment 650 de l'Université Paris-Saclay, plus
précisément au LMF (Laboratoire de Méthodes Formelles)\footnote{LMF.
\url{https://lmf.cnrs.fr/}} dans l'équipe Toccata\footnote{Inria.
\url{https://toccata.gitlabpages.inria.fr/toccata/index.fr.html}}. C'est une
équipe de recherche du centre INRIA Saclay-Île-de-France. Celle-ci est composée
de 7 membres permanents, dont deux responsables : Sylvie Boldo (responsable
permanente) et Claude Marché (responsable scientifique). Parmi eux se trouve
Guillaume Melquiond, notre encadrant de stage. Ses travaux de recherche se
situent à l'intersection des domaines de l'arithmétique informatique et de
la preuve formelle.

  \section{Contexte scientifique}

  Le sujet du stage s'intitule \textit{Formalisation des dictionnaire en Coq}.

    \subsection{Présentation Général}

  Le projet ERC Fresco\footnote{Fresco. \url{https://fresco.gitlabpages.inria.fr/}}
vie à transformer Coq en un outil rapide de calcul algébrique. Un élément clé
est la conception d'un langage de programmation dédié ainsi que des structure de
donnée de haut niveaux. Le but de ce stage est de ce pencher sur les structures
de données associatives, c'est-à-dire les dictionnaires.

  Bien sûr des personnes ont déjà formalisées des dictionnaire en Coq, mais
elles sont toutes faite à l'aide d'arbre binaire, comme les arbres de Patricia
(par exemple dans CompCert) ou des arbres AVL équilibré (dans le module FMap de
la bibliothèque standard de Coq).

    \subsection{Les enjeux}

    \newpage
    \subsection{Travail effectué}

  Le but du stage est d'explorer différentes implémentation de dictionnaire à
base de tableau en Coq. Le premier travaille réalisé était de comprendre
comment sont implémenter les tableau en Coq en lisant la littérature à se sujet.
Pour cela nous avons chercher l'article qui parle de l'apparition de ces tableaux
\cite{armand2010extending}. Nous avons aussi pût trouver des informations sur les
entier machine qui ont été implémenter en Coq en même temps, ils sont
utilisé par les tableaux. En lisant ça j'ai très vite compris que les tableaux
utilisées sont ceux de Baker. Pour compléter le cours au collègue de France de
Xavier Leroy, je suis aller lire l'article de Baker \cite{baker1991shallow} et
celui de Sylvain Conchon, Jean-Christophe Filliâtre \cite{conchon2007persistent}.

    \subsubsection{Implémentations}

  Avant de nous lancer dans l'implémentation des tables de hachage, nous avons
utilisé des dictionnaires basés sur des arbres Patricia afin d'avoir une idée
des spécifications des fonctions de base. Ensuite, en nous renseignant sur les
différentes implémentations des tables de hachage, nous avons pensé qu'il serait
plus simple de commencer par implémenter les tableaux avec une résolution des
conflits à l'aide de listes chaînées.

  Nous avons dû commencer à utiliser les tableaux de Coq et à comprendre leur
interface d'utilisation. Toutes les fonctions de base, telles que
\texttt{get}, \texttt{set}, \texttt{length} et \texttt{copy}, sont disponibles.
De plus, il existe une fonction \texttt{make} qui permet de créer un tableau en
spécifiant un type, une taille et une valeur par défaut pour éviter les
comportements indéfinis.

Ainsi, nous avons défini le type des tables de la manière suivante :

\begin{lstlisting}[language=Coq]
  Inductive bucket : Set :=
    | Empty : bucket
    | Cons : int -> A -> B -> bucket -> bucket.

  Record t : Set := hash_tab {
    size : int;
    hashtab :   PArray.array bucket;
  }.
\end{lstlisting}

  Ensuite, nous avons défini les fonctions de base : \texttt{add}, \texttt{empty},
\texttt{find} et \texttt{findall}. Nous avons de choisir de faire comme les
tables de hachage en Ocaml, c'est à dire de recouvrir les anciennes valeur d'une
clé à chaque ajout d'où l'utilisation du \texttt{findall}. Nous avons donc fait
notre première preuve sur les tableaux en spécifiant ces fonctions.

\begin{lstlisting}[language=Coq]
  Parameter find_spec:
    forall B (ht: t B) key,
    find ht key = List.hd_error (find_all ht key).

  Parameter hempty:
    forall B k s, find (create B s) k = None.

  Parameter add_same:
    forall B k (h: t B) v,
    find_all (add h k v) k = v :: (find_all h k).

  Parameter add_diff:
    forall B k k' (h: t B) v,
    k' <> k -> find_all (add h k v) k' = find_all h k'.
\end{lstlisting}

  Pour effectuer ces preuves, nous avons dû commencer par créer une version non
récursive terminale de la fonction \texttt{findall} afin de pouvoir effectuer
des preuves par calcul dans Coq, ce qui facilite grandement les démonstrations.
Ensuite, nous avons rencontré une première difficulté : un utilisateur pourrait
fournir n'importe quelle table en tant que paramètre, et nous n'avons donc
aucune information sur la valeur par défaut de la table. Deux solutions étaient
possibles : soit établir un invariant sur notre structure de données, soit
ajouter des conditions dans le programme pour gérer les cas impossibles dans
les preuves. Nous avons choisi la deuxième solution, car la création d'un
invariant aurait impliqué la création de termes de preuve très complexes, ce
qui aurait entraîné une perte de la rapidité des entiers machine (il aurait
souvent fallu passer par des représentations de nombres peu performantes
dans les preuves).

  Pour avoir une structure performante, il faut transformer les tableaux en
tableaux dynamique. Nous avons décomposé le problème en 3 fonctions :

\begin{itemize}
  \item \texttt{rehash\_bucket}: replace les élément d'une case pour les mettre
    dans la nouvelle table.
  \item \texttt{copy\_tab}: copie un tableau dans un nouveau tableau.
  \item \texttt{resize}: va redimensionner la table de hachage
\end{itemize}

Nous avons spécifié \texttt{resize} avec cette formule

\begin{lstlisting}[language=Coq]
  Lemma find_all_resize:
    forall (h: t) (k: A),
    find_all (resize h) k = find_all h k.
\end{lstlisting}

  Cette spécification a été très difficile à prouver, car c'est la première
grande preuve que nous avons réalisée entièrement par nous-mêmes. Pour
commencer, nous avons dû attribuer certaines propriétés aux deux autres
fonctions, qui ont été acceptées initialement pour nous assurer de leur utilité
par la suite. Une fois que cette étape de preuve a été réussie, nous avons dû
prouver toutes les assertions liées à la fonction en question, parfois sans
disposer de suffisamment d'hypothèses. Pour mener à bien cette preuve, nous
avons également dû ajouter des conditions dans le programme afin de vérifier
si un élément se trouve effectivement dans le bon compartiment (bucket).

      \subsubsection{Tests}

Pour pouvoir commencer les tests, nous avons ajouté les fonctions manquantes
telles que \texttt{replace} ou \texttt{mem} (Hashtable d'Ocaml
\footnote{Hashtable Ocaml. \url{https://v2.ocaml.org/api/Hashtbl.html}}).
Au départ, nous avons cherché à mettre en place des fonctions qui pourraient
bénéficier d'une optimisation par mémoïsation. J'ai immédiatement pensé à la
fonction de Fibonacci. Cependant, en effectuant cette optimisation, la
complexité devient linéaire et le résultat augmente de manière exponentielle,
ce qui entraîne rapidement des dépassements de capacité.

  \section{Retour d'expérience}

  Ce stage à été une bonne opportunité pour apprendre de nouvelles choses et
en approfondir certaines.

  \section{Remercient}

  \bibliographystyle{alpha}
  \bibliography{ref}

\end{document}
